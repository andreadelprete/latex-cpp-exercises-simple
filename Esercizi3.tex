%%%%%%%%%%%%%%%%%%%%%%%%%%%%%%%%%%%%%%%%%
% Lachaise Assignment
% LaTeX Template
% Version 1.0 (26/6/2018)
%
% This template originates from:
% http://www.LaTeXTemplates.com
%
% Authors:
% Marion Lachaise & François Févotte
% Vel (vel@LaTeXTemplates.com)
%
% License:
% CC BY-NC-SA 3.0 (http://creativecommons.org/licenses/by-nc-sa/3.0/)
% 
%%%%%%%%%%%%%%%%%%%%%%%%%%%%%%%%%%%%%%%%%

%----------------------------------------------------------------------------------------
%	PACKAGES AND OTHER DOCUMENT CONFIGURATIONS
%----------------------------------------------------------------------------------------

\documentclass{article}

%%%%%%%%%%%%%%%%%%%%%%%%%%%%%%%%%%%%%%%%%
% Lachaise Assignment
% Structure Specification File
% Version 1.0 (26/6/2018)
%
% This template originates from:
% http://www.LaTeXTemplates.com
%
% Authors:
% Marion Lachaise & François Févotte
% Vel (vel@LaTeXTemplates.com)
%
% License:
% CC BY-NC-SA 3.0 (http://creativecommons.org/licenses/by-nc-sa/3.0/)
% 
%%%%%%%%%%%%%%%%%%%%%%%%%%%%%%%%%%%%%%%%%

%----------------------------------------------------------------------------------------
%	PACKAGES AND OTHER DOCUMENT CONFIGURATIONS
%----------------------------------------------------------------------------------------

\usepackage{amsmath,amsfonts,stmaryrd,amssymb} % Math packages

\usepackage{hyperref} % For links
\linespread{0.9}

\usepackage{enumerate} % Custom item numbers for enumerations

\usepackage[ruled]{algorithm2e} % Algorithms

\usepackage[framemethod=tikz]{mdframed} % Allows defining custom boxed/framed environments

\usepackage{listings} % File listings, with syntax highlighting
\lstset{
	basicstyle=\ttfamily, % Typeset listings in monospace font
}

%----------------------------------------------------------------------------------------
%	DOCUMENT MARGINS
%----------------------------------------------------------------------------------------

\usepackage{geometry} % Required for adjusting page dimensions and margins

\geometry{
	paper=a4paper, % Paper size, change to letterpaper for US letter size
	top=2.5cm, % Top margin
	bottom=3cm, % Bottom margin
	left=2.5cm, % Left margin
	right=2.5cm, % Right margin
	headheight=14pt, % Header height
	footskip=1.5cm, % Space from the bottom margin to the baseline of the footer
	headsep=1.2cm, % Space from the top margin to the baseline of the header
	%showframe, % Uncomment to show how the type block is set on the page
}

%----------------------------------------------------------------------------------------
%	FONTS
%----------------------------------------------------------------------------------------

\usepackage[utf8]{inputenc} % Required for inputting international characters
\usepackage[T1]{fontenc} % Output font encoding for international characters



%----------------------------------------------------------------------------------------
%	COMMAND LINE ENVIRONMENT
%----------------------------------------------------------------------------------------

% Usage:
% \begin{commandline}
%	\begin{verbatim}
%		$ ls
%		
%		Applications	Desktop	...
%	\end{verbatim}
% \end{commandline}

\mdfdefinestyle{commandline}{
	leftmargin=10pt,
	rightmargin=10pt,
	innerleftmargin=15pt,
	middlelinecolor=black!50!white,
	middlelinewidth=2pt,
	frametitlerule=false,
	backgroundcolor=black!5!white,
	frametitle={Command Line},
	frametitlefont={\normalfont\sffamily\color{white}\hspace{-1em}},
	frametitlebackgroundcolor=black!50!white,
	nobreak,
}

% Define a custom environment for command-line snapshots
\newenvironment{commandline}{
	\medskip
	\begin{mdframed}[style=commandline]
}{
	\end{mdframed}
	\medskip
}

%----------------------------------------------------------------------------------------
%	FILE CONTENTS ENVIRONMENT
%----------------------------------------------------------------------------------------

% Usage:
% \begin{file}[optional filename, defaults to "File"]
%	File contents, for example, with a listings environment
% \end{file}

\mdfdefinestyle{file}{
	innertopmargin=1.6\baselineskip,
	innerbottommargin=0.8\baselineskip,
	topline=false, bottomline=false,
	leftline=false, rightline=false,
	leftmargin=2cm,
	rightmargin=2cm,
	singleextra={%
		\draw[fill=black!10!white](P)++(0,-1.2em)rectangle(P-|O);
		\node[anchor=north west]
		at(P-|O){\ttfamily\mdfilename};
		%
		\def\l{3em}
		\draw(O-|P)++(-\l,0)--++(\l,\l)--(P)--(P-|O)--(O)--cycle;
		\draw(O-|P)++(-\l,0)--++(0,\l)--++(\l,0);
	},
	nobreak,
}

% Define a custom environment for file contents
\newenvironment{file}[1][File]{ % Set the default filename to "File"
	\medskip
	\newcommand{\mdfilename}{#1}
	\begin{mdframed}[style=file]
}{
	\end{mdframed}
	\medskip
}

%----------------------------------------------------------------------------------------
%	NUMBERED QUESTIONS ENVIRONMENT
%----------------------------------------------------------------------------------------

% Usage:
% \begin{question}[optional title]
%	Question contents
% \end{question}

\mdfdefinestyle{question}{
	innertopmargin=1.2\baselineskip,
	innerbottommargin=0.8\baselineskip,
	roundcorner=5pt,
	nobreak,
	singleextra={%
		\draw(P-|O)node[xshift=1em,anchor=west,fill=white,draw,rounded corners=5pt]{%
		Question \theQuestion\questionTitle};
	},
}

\newcounter{Question} % Stores the current question number that gets iterated with each new question

% Define a custom environment for numbered questions
\newenvironment{question}[1][\unskip]{
	\bigskip
	\stepcounter{Question}
	\newcommand{\questionTitle}{~#1}
	\begin{mdframed}[style=question]
}{
	\end{mdframed}
	\medskip
}

%----------------------------------------------------------------------------------------
%	WARNING TEXT ENVIRONMENT
%----------------------------------------------------------------------------------------

% Usage:
% \begin{warn}[optional title, defaults to "Warning:"]
%	Contents
% \end{warn}

\mdfdefinestyle{warning}{
	topline=false, bottomline=false,
	leftline=false, rightline=false,
	nobreak,
	singleextra={%
		\draw(P-|O)++(-0.5em,0)node(tmp1){};
		\draw(P-|O)++(0.5em,0)node(tmp2){};
		\fill[black,rotate around={45:(P-|O)}](tmp1)rectangle(tmp2);
		\node at(P-|O){\color{white}\scriptsize\bf !};
		\draw[very thick](P-|O)++(0,-1em)--(O);%--(O-|P);
	}
}

% Define a custom environment for warning text
\newenvironment{warn}[1][Warning:]{ % Set the default warning to "Warning:"
	\medskip
	\begin{mdframed}[style=warning]
		\noindent{\textbf{#1}}
}{
	\end{mdframed}
}

%----------------------------------------------------------------------------------------
%	INFORMATION ENVIRONMENT
%----------------------------------------------------------------------------------------

% Usage:
% \begin{info}[optional title, defaults to "Info:"]
% 	contents
% 	\end{info}

\mdfdefinestyle{info}{%
	topline=false, bottomline=false,
	leftline=false, rightline=false,
	nobreak,
	singleextra={%
		\fill[black](P-|O)circle[radius=0.4em];
		\node at(P-|O){\color{white}\scriptsize\bf i};
		\draw[very thick](P-|O)++(0,-0.8em)--(O);%--(O-|P);
	}
}

% Define a custom environment for information
\newenvironment{info}[1][Info:]{ % Set the default title to "Info:"
	\medskip
	\begin{mdframed}[style=info]
		\noindent{\textbf{#1}}
}{
	\end{mdframed}
}

% Disable paragraph indentation
%\setlength{\parindent}{0pt}
 % Include the file specifying the document structure and custom commands

%----------------------------------------------------------------------------------------
%	ASSIGNMENT INFORMATION
%----------------------------------------------------------------------------------------

\title{Esercizi Blocco 3} % Title of the assignment

\author{Luca Oliveri\\ \texttt{luca.olivieri-1@unitn.it}} % Author name and email address

\date{Università di Trento --- \today} % University, school and/or department name(s) and a date

%----------------------------------------------------------------------------------------

\begin{document}

\maketitle % Print the title

\section*{Introduzione} % Unnumbered section

\setcounter{section}{3}
Focus principale su manipolazione di stringhe. Usare quando possibile le funzioni della \href{http://www.cplusplus.com/reference/cstring/}{\texttt{cstring}} e \href{http://www.cplusplus.com/reference/cctype/}{\texttt{ctype}}. Essendo che alcuni di questi esercizi richiedono di essere risolti dividendo il problema principale in più problemi semplici e puntuali, dedicare tempo ad organizzare il codice in modo adeguato.

%------------------------------------------------

\subsection{}
SUP che prende una stringa del tipo \texttt{"923D"} ed estrae il numero, convertendolo in \texttt{int}. La lettera D è sempre alla fine della stringa. Verificare il funzionamento anche con numeri negativi. Esistono \href{https://www.geeksforgeeks.org/converting-strings-numbers-cc/}{diversi modi} per fare questa conversione, ma probabilmente la soluzione più semplice è usare il metodo nativo C++, che è la prima soluzione proposta dal sito.

\subsection{}
SUP che prende una stringa dall'utente che può essere anche una frase, comprensiva quindi di spazi e punteggiatura. Il programma ristampa la stessa stringa al contrario.  

\subsection{}
Localizzare in una stringa tutte le occorrenze di un determinato carattere usando la funzione di libreria \texttt{strchr}. Le stringhe in ingresso sono composte di sole lettere minuscole e spazi. Il programma stampa la stessa stringa in ingresso con le occorrenze trovate convertite a lettere maiuscole (cercare funzione di libreria per fare ciò). Stampa inoltre un conteggio delle occorrenze.

\subsection{}
SUP che usa l'abilità acquisita nell'esercizio precedente di fare \textit{parsing} di stringa e svolge una funzionalità programmabile. Il programma prende in input una stringa di valori di temperatura del tipo \texttt{"20C@34F@12F@23C"} e converte questi in gradi Kelvin. 

% Stampare due tabelle distinte per conversioni Celsius e Fahrenheit.

\subsection{}
SUP che data una stringa in input calcola le occorrenze di ogni carattere. Risolvere il problema usando un unico array per contare le occorrenze. Stampare il risultato in tabella, con possibili interessanti statistiche a vostra discrezione. Considerare solo lettere dell'alfabeto, maiuscole e minuscole fanno parte dello stesso conteggio. Scartare tutto ciò che non è una lettera, per semplicità si scartano anche i caratteri della tabella ASCII estesa come le lettere accentate.
\begin{info}
	Tenere a mente come sono rappresentate le lettere nella tabella ASCII. È possibile indicizzare l'array dove incrementiamo mano a mano il conteggio usando direttamente la lettera sotto esame e applicando un offset. Ad esempio la lettera D, quarta nell'alfabeto quindi avente indice 3, ha valore ASCII 44, oppure anche 64 nel nostro caso. Sottraendo un offset di 41 o di 61 rispettivamente, troviamo il nostro indice di valore 3.
\end{info}
\begin{warn} [Estensione avanzata:]
	Invece che prendere in input una stringa dell'utente, prendere in input il nome di un file di testo \textit{plain-text}, in altre parole un file \texttt{.txt}. È possibile trovare online interi testi in formato \texttt{.txt}, come ad esempio \href{http://rosada.yolasite.com/resources/i_promessi%20spo unsi.txt}{I Promessi Sposi}, oppure anche \href{https://www.google.com/url?sa=t&rct=j&q=&esrc=s&source=web&cd=&ved=2ahUKEwjEuuHktontAhWQa8AKHdv_D3AQFjAAegQIBBAC&url=http%3A%2F%2Fwww.hoepliscuola.it%2Fdownload%2F2842%2Fla-divina-commedia-txt.aspx&usg=AOvVaw2-ivHQ7MdppenKh_hOjUeB}{La Divina Commedia}. La manipolazione di file in C++ usa lo stesso paradigma di uso della console. Trovate \href{http://www.cplusplus.com/reference/istream/istream/get/}{qui} quello che vi serve specificatamente per risolvere questo esercizio, in particolare basatevi sull'esempio.
\end{warn}

\subsection{}
SUP che simula il comportamento di due dadi. I dadi possono avere numero di facce variabile, ma per verificare la correttezza del programma cominciare con dadi a sei facce. Il programma 'tira' i due dadi usando la funzione \texttt{rand}, opportunamente inizializzata. La somma dei valori di ogni lancio è salvata in array. Il programma deve lanciare i dadi decine di migliaia di volte, numero specificato dall'utente, e stampare la probabilità di ogni valore sotto forma di tabella. Verificare che somma 7 è la più probabile con probabilità di un sesto, essendo che ci sono 6 lanci diversi che restituiscono somma 7. Indagare poi con dadi aventi numero di facce diverse da sei.
\begin{info} 
	Il programma può simulare qualsiasi tipo di dado, ma in realtà solo alcuni poligono possono richiudersi in un solido. \href{https://en.wikipedia.org/wiki/Dice#Variants}{Qua} una lista dei possibili dadi, dai più classici ai più esoterici.
\end{info}



\section*{Esercizi \texttt{CodeStepByStep}}
\begin{itemize}
	\item \href{https://www.codestepbystep.com/problem/view/cpp/algorithms/isPermutation}{isPermutation}
	\item \href{https://www.codestepbystep.com/problem/view/cpp/algorithms/longestUniqueString}{longestUniqueString}
	\item \href{https://www.codestepbystep.com/problem/view/cpp/algorithms/compressString}{compressString}
\end{itemize}




\end{document}
