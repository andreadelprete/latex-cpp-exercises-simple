%%%%%%%%%%%%%%%%%%%%%%%%%%%%%%%%%%%%%%%%%
% Lachaise Assignment
% LaTeX Template
% Version 1.0 (26/6/2018)
%
% This template originates from:
% http://www.LaTeXTemplates.com
%
% Authors:
% Marion Lachaise & François Févotte
% Vel (vel@LaTeXTemplates.com)
%
% License:
% CC BY-NC-SA 3.0 (http://creativecommons.org/licenses/by-nc-sa/3.0/)
% 
%%%%%%%%%%%%%%%%%%%%%%%%%%%%%%%%%%%%%%%%%

%----------------------------------------------------------------------------------------
%	PACKAGES AND OTHER DOCUMENT CONFIGURATIONS
%----------------------------------------------------------------------------------------

\documentclass{article}

\input{../../structure.tex} % Include the file specifying the document structure and custom commands

%----------------------------------------------------------------------------------------
%	ASSIGNMENT INFORMATION
%----------------------------------------------------------------------------------------

\title{Esercizi Blocco 3} % Title of the assignment

\author{Luca Oliveri \& Andrea Del Prete\\ \texttt{olivieri.luca@outlook.com}, \texttt{andrea.delprete@unitn.it}} % Author name and email address

\date{Universit� di Trento --- \today} % University, school and/or department name(s) and a date

%----------------------------------------------------------------------------------------

\begin{document}

\maketitle % Print the title

\section*{Introduzione} % Unnumbered section

\setcounter{section}{3}
Questi esercizi si concentrano sulla manipolazione di stringhe. Usare quando possibile le funzioni della \href{http://www.cplusplus.com/reference/cstring/}{\texttt{cstring}} e \href{http://www.cplusplus.com/reference/cctype/}{\texttt{ctype}}. Essendo che alcuni di questi esercizi richiedono di essere risolti dividendo il problema principale in pi� problemi semplici e puntuali, dedicare tempo ad organizzare il codice in modo adeguato.

%------------------------------------------------

\subsection{}
SUP che legge una stringa del tipo \texttt{"923D"} e ne estrae la parte numerica, convertendola in \texttt{int}. La lettera D � sempre alla fine della stringa. Verificare il funzionamento anche con numeri negativi. Esistono \href{https://www.geeksforgeeks.org/converting-strings-numbers-cc/}{diversi modi} per fare questa conversione, ma probabilmente la soluzione pi� semplice � usare il metodo nativo C++, che � la prima soluzione proposta dal sito.

\subsection{}
SUP che prende una stringa dall'utente che pu� essere anche una frase, comprensiva quindi di spazi e punteggiatura. Il programma ristampa la stessa stringa al contrario.  

\subsection{}
SUP che dati in ingresso una stringa ed un carattere, stampa la stessa stringa con le occorrenze trovate convertite a lettere maiuscole. Stampa inoltre un conteggio delle occorrenze.

\begin{info}
Ci sono diverse funzioni di libreria che possono essere utili in questo esercizio. Consultare \href{http://www.cplusplus.com/reference/string/string/}{l'elenco} completo e scegliere quella ritenuta pi� adatta.
\end{info}

\subsection{}
SUP che esegue \textit{parsing} di stringa e svolge una funzionalit� programmabile. Nello specifico il programma prende in ingresso una stringa di valori di temperatura del tipo \texttt{"20C@34F@12F@23C"} e li converte in gradi Kelvin. Stampare due tabelle distinte per conversioni Celsius e Fahrenheit. La stringa da elaborare pu� essere scritta direttamente nel codice o passata come argomento da riga di comando.
\begin{info}
	Come nell'esercizio precedente, diverse funzioni di libreria possono aiutare a dividere la stringa in pi� parti. In questo la scelta pi� adatta � \texttt{strtok}.
\end{info}

\clearpage

\subsection{}
SUP che data una stringa in input calcola le occorrenze di ogni carattere. Risolvere il problema usando un unico array per contare le occorrenze. Stampare il risultato in tabella, con statistiche a vostra discrezione. Considerare solo lettere dell'alfabeto. Maiuscole e minuscole devono fare parte dello stesso conteggio. Scartare tutto ci� che non � una lettera. Per semplicit� scartare anche i caratteri della tabella ASCII estesa come le lettere accentate.
\begin{info}
Tenere a mente come sono rappresentate le lettere nella tabella ASCII. � possibile indicizzare l'array dove incrementiamo mano a mano il conteggio usando direttamente la lettera sotto esame e applicando un offset. Ad esempio la lettera `D`, quarta nell'alfabeto, quindi avente indice 3, ha valore ASCII 44, oppure anche 64 nel nostro caso. Sottraendo un offset di 41 o di 61 rispettivamente, troviamo il nostro indice di valore 3.
\end{info}
\begin{warn} [Estensione avanzata:]
Invece di prendere in ingresso una stringa dell'utente, prendere in ingresso il nome di un file di testo \textit{plain-text}, in altre parole un file \texttt{.txt}. � possibile trovare online interi testi in formato \texttt{.txt}, come ad esempio \href{http://rosada.yolasite.com/resources/i_promessi\%20sposi.txt}{I Promessi Sposi}, oppure anche \href{https://raw.githubusercontent.com/dlang/druntime/master/benchmark/extra-files/dante.txt}{La Divina Commedia}. La manipolazione di file in C++ usa lo stesso paradigma di uso della console (\texttt{cout, cin}). Trovate \href{http://www.cplusplus.com/reference/istream/istream/get/}{qui} quello che vi serve specificatamente per risolvere questo esercizio.
\end{warn}

\subsection{}
SUP che simula il comportamento di due dadi. I dadi possono avere un numero di facce variabile, ma per verificare la correttezza del programma cominciare con dadi a sei facce. Il programma 'tira' i due dadi usando la funzione \texttt{rand}, opportunamente inizializzata. La somma dei valori di ogni lancio � salvata in un array. Il programma deve lanciare i dadi decine di migliaia di volte e stampare la probabilit�di ogni valore sotto forma di tabella. Verificare che il risultato 7 � il pi� probabile, con probabilit� di un sesto, essendo che ci sono 6 combinazioni diverse che restituiscono somma 7. Verificare anche come aumentando il numero di lanci le probabilit� tendano a diventare simmetriche rispetto al valore centrale, descrivendo una curva Gaussiana. Indagare poi con dadi aventi numero di facce diverse da sei. Sia il numero di facce che il numero di lanci sono inseriti dall'utente all'avvio del programma.
\begin{info} 
	Il programma pu� simulare qualsiasi tipo di dado, ma in realt�solo alcuni poligoni possono richiudersi in un solido. \href{https://en.wikipedia.org/wiki/Dice#Variants}{Qui} una lista dei possibili dadi, dai pi� classici ai pi� esoterici.
\end{info}

\section*{Esercizi \texttt{CodeStepByStep}}
\begin{itemize}
	\item \href{https://www.codestepbystep.com/problem/view/cpp/algorithms/isPermutation}{isPermutation}
	\item \href{https://www.codestepbystep.com/problem/view/cpp/algorithms/longestUniqueString}{longestUniqueString}
	\item \href{https://www.codestepbystep.com/problem/view/cpp/algorithms/compressString}{compressString}
\end{itemize}


\end{document}
