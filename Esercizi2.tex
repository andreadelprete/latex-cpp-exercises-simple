%%%%%%%%%%%%%%%%%%%%%%%%%%%%%%%%%%%%%%%%%
% Lachaise Assignment
% LaTeX Template
% Version 1.0 (26/6/2018)
%
% This template originates from:
% http://www.LaTeXTemplates.com
%
% Authors:
% Marion Lachaise & François Févotte
% Vel (vel@LaTeXTemplates.com)
%
% License:
% CC BY-NC-SA 3.0 (http://creativecommons.org/licenses/by-nc-sa/3.0/)
% 
%%%%%%%%%%%%%%%%%%%%%%%%%%%%%%%%%%%%%%%%%

%----------------------------------------------------------------------------------------
%	PACKAGES AND OTHER DOCUMENT CONFIGURATIONS
%----------------------------------------------------------------------------------------

\documentclass{article}

\input{structure.tex} % Include the file specifying the document structure and custom commands

%----------------------------------------------------------------------------------------
%	ASSIGNMENT INFORMATION
%----------------------------------------------------------------------------------------

\title{Esercizi Blocco 1} % Title of the assignment

\author{Luca Oliveri\\ \texttt{luca.olivieri-1@unitn.it}} % Author name and email address

\date{Università di Trento --- \today} % University, school and/or department name(s) and a date

%----------------------------------------------------------------------------------------

\begin{document}

\maketitle % Print the title

\section*{Introduzione} % Unnumbered section
Questi esercizi richiedono aver affrontato l'argomento array. Quando si chiede di estendere le funzionalità di un altro programma si consiglia di non modificare quello vecchio, ma di iniziarne uno nuovo.

\setcounter{section}{2}

%------------------------------------------------

\subsection{}
SUP che legge 10 numeri dall'utente e ne calcola la somma. Fermarsi a leggere numeri quando l'utente inserisce tutti i 10 numeri oppure quando inserisce 0.

\subsection{}
SUP che prende 10 numeri e calcola la somma solamente dei numeri pari.

\subsection{}
SUP che calcola la media dei voti di un esame. Il programma si fa dare prima il numero totale di studenti, poi i voti. Supporre un limite massimo di studenti. 

\begin{info} 
	Il limite massimo dipende da quanto grandi decidiamo che siano le strutture in fase di compilazione. Impareremo che è possibile scrivere programmi che non hanno questo tipo di limite.
\end{info}

\subsection{}
Trovare il numero più grande in un array è un problema ricorrente in programmazione. SUP che dati 10 numeri, inseriti dall'utente, restituisce il numero più grande.

\subsection{}
Modificare il programma precedente in modo che trovi i due numeri più grandi.

\subsection{}
Estendere il programma 1.2 perché l'utente possa inserire numeri a piacimento numeri lunghi fino a 12 cifre.






\section*{Esercizi \texttt{CodeStepByStep}}
\begin{itemize}
	\item 
\end{itemize}



\end{document}
