%%%%%%%%%%%%%%%%%%%%%%%%%%%%%%%%%%%%%%%%%
% Lachaise Assignment
% LaTeX Template
% Version 1.0 (26/6/2018)
%
% This template originates from:
% http://www.LaTeXTemplates.com
%
% Authors:
% Marion Lachaise & François Févotte
% Vel (vel@LaTeXTemplates.com)
%
% License:
% CC BY-NC-SA 3.0 (http://creativecommons.org/licenses/by-nc-sa/3.0/)
% 
%%%%%%%%%%%%%%%%%%%%%%%%%%%%%%%%%%%%%%%%%

%----------------------------------------------------------------------------------------
%	PACKAGES AND OTHER DOCUMENT CONFIGURATIONS
%----------------------------------------------------------------------------------------

\documentclass{article}

\input{../../structure.tex} % Include the file specifying the document structure and custom commands

%----------------------------------------------------------------------------------------
%	ASSIGNMENT INFORMATION
%----------------------------------------------------------------------------------------

\title{Esercizi Blocco 2} % Title of the assignment

\author{Luca Oliveri \& Andrea Del Prete\\ \texttt{olivieri.luca@outlook.com}, \texttt{andrea.delprete@unitn.it}} % Author name and email address

\date{Universit� di Trento --- \today} % University, school and/or department name(s) and a date

%----------------------------------------------------------------------------------------

\begin{document}

\maketitle % Print the title

\section*{Introduzione} % Unnumbered section
Molti di questi esercizi (ma non tutti) richiedono aver affrontato l'argomento array. Quando si chiede di estendere le funzionalit� di un altro programma si consiglia di non modificare quello vecchio, ma di iniziarne uno nuovo. Si consiglia di usare le funzioni quando appropriato per rendere il codice pi� leggibile e modulare.

\setcounter{section}{2}

%------------------------------------------------

\subsection{}
SUP che legge 10 numeri e ne calcola la somma. Fermare la lettura dei numeri quando l'utente ha inserito tutti i 10 numeri oppure quando inserisce 0. A fine inserimento il programma stampa su due colonne il numero inserito e l'ultimo valore di somma.
\begin{info} 
	Il limite massimo dipende da quanto grandi decidiamo che siano le strutture in fase di compilazione. Impareremo che � possibile scrivere programmi che non hanno questo limite.
\end{info}

\subsection{}
Riscrivere l'esercizio 1.3 usando gli array, con lunghezza variabile delle cifre del numero da verificare se palindromo.

\subsection{}
Modificare il programma 1.3 perch� l'utente possa inserire frasi anzich� numeri, lunghe fino a 100 caratteri. 
\begin{info}
	\href{https://www2.cs.arizona.edu/icon/oddsends/palinsen.htm}{Qui una lista} di frasi palindrome con cui fare dei test, tenendo a mente che la stessa lettera maiuscola e minuscola ha due codici ASCII diversi. Inoltre in queste frasi la punteggiatura non � palindroma. Usare \href{http://www.cplusplus.com/reference/cctype/tolower/}{questo} e \href{http://www.cplusplus.com/reference/cctype/ispunct/}{questo} per modificare la stringa inserita dall'utente rendendo tutte le lettere minuscole e togliendo la punteggiatura.
\end{info}

\subsection{}
SUP che legge diversi numeri decimali e successivamente li ristampa approssimati all'intero pi� vicino. Usare la funzione di libreria \texttt{round}. Stampare i numeri come interi, quindi, ad esempio, non stampare \texttt{42.0} ma \texttt{42}.

\subsection{}
SUP che calcola l'ipotenusa di un triangolo dati i due cateti. 

\subsection{}
SUP che stampa una tabella ben formattata con le scale Fahrenheit e Celsius, da 247 a 250 gradi Kelvin, a intervalli di 0.25 gradi. 
Il programma deve usare due funzioni di conversione della temperatura, da Kelvin a Celsius, e da Kelvin a Fahrenheit. 

\subsection{}
SUP che calcola la media dei voti di un esame. Il programma legge prima il numero totale di studenti, poi i voti. Supporre un limite massimo di studenti. Approssimare il risultato all'intero pi� vicino. 

\subsection{}
Trovare il numero pi� grande in un array � un problema ricorrente in programmazione. SUP che dati 10 numeri restituisce il numero pi� grande. I numeri sono inizializzati direttamente insieme alla dichiarazione dell'array. 

\subsection{}
Modificare il programma precedente in modo che trovi i due numeri (distinti) pi� grandi.
Se il numero pi� grande � presente due volte, una occorrenza deve essere trascurata nel cercare il secondo maggiore. 

\subsection{}
Modificare il programma dei numeri primi cambiando il limite superiore della ricerca di divisori perfetti con la radice del numero sotto esame, anzich� la sua met�. Confrontare l'output del nuovo algoritmo con quello vecchio. 
\begin{info} 
Cercare online come si pu� temporizzare l'esecuzione di un programma e stimare il guadagno in velocit� con questa modifica.
\end{info}

\subsection{}
Un intero � detto \emph{perfetto} quando la somma dei suoi fattori restituisce il numero stesso. Ad esempio 6 � un numero \emph{perfetto} essendo che $1 + 2 + 3 = 6$. 
Scrivere una funzione che determina se un numero � \emph{perfetto}. Usarla per ricavare tutti i numeri \emph{perfetti} da 1 a 10'000. 
\begin{info} 
Se siete riusciti a temporizzare il programma precedente, provate ad alzare il limite della ricerca e confrontare i tempi di calcolo con i vostri compagni.
\end{info}

\subsection{}
SUP che sceglie un numero casuale tra 1 e 1000 usando la \href{http://www.cplusplus.com/reference/cstdlib/rand/}{funzione} \texttt{rand}. 
Il programma poi legge numeri inseriti dall'utente fintanto che uno di questi � uguale al numero estratto. 
Il programma guida l'utente nella sua ricerca indicando se l'ultimo numero inserito � maggiore o minore del numero estratto. � una ricerca di tipo binario.

\subsection{}
SUP che legge 20 numeri e li ristampa ordinati in ordine ascendente. Usare due cicli innestati, dove quello pi� interno confronta due numeri adiacenti, scambiandoli se fuori ordine. Il ciclo esterno continua fintanto che durante un ciclo completo non avvengono pi� scambi, quando quindi tutti i numeri sono in ordine.

\begin{warn}
Modificare questo esercizio quando sarete in grado di scrivere funzioni che prendono in input un array. Nel frattempo risolvete il problema nel main.
\end{warn}
\begin{info}
Riordinare un array, o \textit{sorting}, � un problema estremamente ricorrente in programmazione. Per questo motivo � anche un soggetto di ricerca ed esistono \href{https://en.wikipedia.org/wiki/Sorting_algorithm}{molti} algoritmi diversi. In \href{https://www.youtube.com/watch?v=kPRA0W1kECg}{questo video} una bella visualizzazione mette a confronto come diversi algoritmi riordinano un array. Notare come la dimensione degli array sia diversa (colonnine pi� fitte) ma il tempo di esecuzione di ciascun algoritmo � simile. Nell'esercizio andiamo ad implementare il \emph{Bubble Sort}, uno degli algoritmi pi� lenti, ma probabilmente il pi� semplice. Un esercizio avanzato, ma interessante, � di implementare un altro algoritmo di sorting tra quelli della lista su Wikipedia e confrontare i tempi di calcolo usando array di grandi dimensioni riempiti con numeri casuali. 
\end{info}

\section*{Esercizi \texttt{CodeStepByStep}}
\href{https://www.codestepbystep.com/problem/list/cpp}{Tutti gli esercizi} della categoria \texttt{array} saranno affrontabili solamente quando saprete come scrivere funzioni che prendono come parametro un array.

\end{document}
