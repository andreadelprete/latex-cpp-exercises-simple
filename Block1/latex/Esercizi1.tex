%%%%%%%%%%%%%%%%%%%%%%%%%%%%%%%%%%%%%%%%%
% Lachaise Assignment
% LaTeX Template
% Version 1.0 (26/6/2018)
%
% This template originates from:
% http://www.LaTeXTemplates.com
%
% Authors:
% Marion Lachaise & François Févotte
% Vel (vel@LaTeXTemplates.com)
%
% License:
% CC BY-NC-SA 3.0 (http://creativecommons.org/licenses/by-nc-sa/3.0/)
% 
%%%%%%%%%%%%%%%%%%%%%%%%%%%%%%%%%%%%%%%%%

%----------------------------------------------------------------------------------------
%	PACKAGES AND OTHER DOCUMENT CONFIGURATIONS
%----------------------------------------------------------------------------------------

\documentclass{article}

\input{../../structure.tex} % Include the file specifying the document structure and custom commands

%----------------------------------------------------------------------------------------
%	ASSIGNMENT INFORMATION
%----------------------------------------------------------------------------------------

\title{Esercizi Blocco 1} % Title of the assignment

\author{Luca Oliveri \& Andrea Del Prete\\ \texttt{olivieri.luca@outlook.com}, \texttt{andrea.delprete@unitn.it}} % Author name and email address
\date{Universit� di Trento --- \today} % University, school and/or department name(s) and a date

%----------------------------------------------------------------------------------------

\begin{document}

\maketitle % Print the title

%----------------------------------------------------------------------------------------
%	INTRODUCTION
%----------------------------------------------------------------------------------------

\section*{Introduzione} % Unnumbered section
Abbreviamo "scrivere un programma" con SUP. Questo blocco prevede l'uso di funzioni. Non � esplicitamente indicato quali porzioni di codice sono da delegare ad una funzione.


\setcounter{section}{1}

%------------------------------------------------

\subsection{} 
SUP che legge due numeri interi $x$ e $y$ e calcola $x^y$. Implementare due versioni, una che per il loop interno usa \texttt{for}, l'altra che usa \texttt{while}. Non usare \texttt{pow()}, questo perch� prende in ingresso due \texttt{double}, quindi compie un elevamento a potenza tra numeri reali. Pur permettendoci di ottenere lo stesso risultato, l'esecuzione sar� molto pi� lenta e inoltre potrebbe risultare in errori numerici.

\subsection{}
SUP che dato un numero intero $x$ e un numero indice $i$, individua la cifra alla posizione $i$ del numero $x$. Esempio: $x=39842$, indice $i=2$, risultato 4.
\begin{info} 
Per scomporre il numero in cifre, usare l'operatore divisione e resto con le potenze di 10. Questa � un'operazione comune in programmazione. Svolgere prima l'esercizio precedente: si pu� riutilizzare del codice.
\end{info}

\subsection{}
Una parola, frase o numero palindromo ha la propriet� che pu� essere letto in entrambi i sensi. Per esempio i seguenti numeri da 5 cifre sono palindromi: 12321, 55555, 85658, 11611. SUP che legge un numero a 5 cifre e determina se questo � palindromo.
\begin{info} 
Si consiglia di svolgere prima i due esercizi precedenti.
\end{info}
\begin{warn} [Estensione:]
Il numero in ingresso non ha lunghezza fissa di 5, ma ha lunghezza variabile. Con questa estensione diventa l'esercizio algoritmicamente pi� complesso di questo Blocco.
\end{warn}

\subsection{}
SUP che stampa senza sosta i multipli di due. Quindi 2, 4, 8, 16, eccetera. Il loop � infinito. Cosa succede e perch�?

\subsection{}
Il fattoriale di un numero non-negativo $n$ si scrive $n!$. Ed � definito come segue:
$$n! = n \cdot (n - 1) \cdot (n - 2) \cdot (n - 3) \cdot \ldots \cdot 1$$
Se $n==0$ allora $0! = 1$.
Per esempio $$5! = 5 \cdot 4 \cdot 3 \cdot 2 \cdot 1 = 120$$
SUP che legge un numero non-negativo e ne calcola il fattoriale. Fino a che numero si ottengono risultati corretti? Controllare che l'utente inserisca un numero adeguato e in caso contrario scrivere un messaggio e terminare l'esecuzione. 
\begin{info} 
\href{https://en.wikipedia.org/wiki/Recursion_(computer_science)#Recursion_versus_iteration}{� dimostrato} che algoritmi scritti usando la ricorsione possono essere riscritti usando cicli iterativi classici (\texttt{while}/\texttt{for}) e viceversa. Provare a scrivere entrambe le soluzioni.
\end{info}


\subsection{}
SUP che prende in ingresso un intero e calcola il valore della costante di Eulero $e$ con la seguente formula:
$$e = 1 + \frac{1}{1!} + \frac{1}{2!} + \frac{1}{3!} + \cdots$$
Usare il valore in ingresso per determinare a quale termine troncare il calcolo della progressione. Confrontare il valore con la funzione \texttt{exp()} presente in \texttt{math.h} con argomento 1. Se avete dubbi sulle funzioni di libreria, Google � \href{http://www.cplusplus.com/reference/cmath/exp/}{tuo} \href{https://www.geeksforgeeks.org/exp-function-cpp/}{amico}. Quanti termini sono necessari per far scendere l'errore sotto ai millesimi?

\section*{Esercizi \texttt{CodeStepByStep}}
\begin{itemize}
	\item \href{https://www.codestepbystep.com/problem/view/cpp/parameters/BMI}{BMI}
	\item \href{https://www.codestepbystep.com/problem/view/cpp/parameters/boxOfStars}{boxOfStars}
	\item \href{https://www.codestepbystep.com/problem/view/cpp/parameters/Projectile}{Projectile}
\end{itemize}



\end{document}
