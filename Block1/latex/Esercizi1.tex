%%%%%%%%%%%%%%%%%%%%%%%%%%%%%%%%%%%%%%%%%
% Lachaise Assignment
% LaTeX Template
% Version 1.0 (26/6/2018)
%
% This template originates from:
% http://www.LaTeXTemplates.com
%
% Authors:
% Marion Lachaise & François Févotte
% Vel (vel@LaTeXTemplates.com)
%
% License:
% CC BY-NC-SA 3.0 (http://creativecommons.org/licenses/by-nc-sa/3.0/)
% 
%%%%%%%%%%%%%%%%%%%%%%%%%%%%%%%%%%%%%%%%%

%----------------------------------------------------------------------------------------
%	PACKAGES AND OTHER DOCUMENT CONFIGURATIONS
%----------------------------------------------------------------------------------------

\documentclass{article}

%%%%%%%%%%%%%%%%%%%%%%%%%%%%%%%%%%%%%%%%%
% Lachaise Assignment
% Structure Specification File
% Version 1.0 (26/6/2018)
%
% This template originates from:
% http://www.LaTeXTemplates.com
%
% Authors:
% Marion Lachaise & François Févotte
% Vel (vel@LaTeXTemplates.com)
%
% License:
% CC BY-NC-SA 3.0 (http://creativecommons.org/licenses/by-nc-sa/3.0/)
% 
%%%%%%%%%%%%%%%%%%%%%%%%%%%%%%%%%%%%%%%%%

%----------------------------------------------------------------------------------------
%	PACKAGES AND OTHER DOCUMENT CONFIGURATIONS
%----------------------------------------------------------------------------------------

\usepackage{amsmath,amsfonts,stmaryrd,amssymb} % Math packages

\usepackage{hyperref} % For links
\linespread{0.9}

\usepackage{enumerate} % Custom item numbers for enumerations

\usepackage[ruled]{algorithm2e} % Algorithms

\usepackage[framemethod=tikz]{mdframed} % Allows defining custom boxed/framed environments

\usepackage{listings} % File listings, with syntax highlighting
\lstset{
	basicstyle=\ttfamily, % Typeset listings in monospace font
}

%----------------------------------------------------------------------------------------
%	DOCUMENT MARGINS
%----------------------------------------------------------------------------------------

\usepackage{geometry} % Required for adjusting page dimensions and margins

\geometry{
	paper=a4paper, % Paper size, change to letterpaper for US letter size
	top=2.5cm, % Top margin
	bottom=3cm, % Bottom margin
	left=2.5cm, % Left margin
	right=2.5cm, % Right margin
	headheight=14pt, % Header height
	footskip=1.5cm, % Space from the bottom margin to the baseline of the footer
	headsep=1.2cm, % Space from the top margin to the baseline of the header
	%showframe, % Uncomment to show how the type block is set on the page
}

%----------------------------------------------------------------------------------------
%	FONTS
%----------------------------------------------------------------------------------------

\usepackage[utf8]{inputenc} % Required for inputting international characters
\usepackage[T1]{fontenc} % Output font encoding for international characters



%----------------------------------------------------------------------------------------
%	COMMAND LINE ENVIRONMENT
%----------------------------------------------------------------------------------------

% Usage:
% \begin{commandline}
%	\begin{verbatim}
%		$ ls
%		
%		Applications	Desktop	...
%	\end{verbatim}
% \end{commandline}

\mdfdefinestyle{commandline}{
	leftmargin=10pt,
	rightmargin=10pt,
	innerleftmargin=15pt,
	middlelinecolor=black!50!white,
	middlelinewidth=2pt,
	frametitlerule=false,
	backgroundcolor=black!5!white,
	frametitle={Command Line},
	frametitlefont={\normalfont\sffamily\color{white}\hspace{-1em}},
	frametitlebackgroundcolor=black!50!white,
	nobreak,
}

% Define a custom environment for command-line snapshots
\newenvironment{commandline}{
	\medskip
	\begin{mdframed}[style=commandline]
}{
	\end{mdframed}
	\medskip
}

%----------------------------------------------------------------------------------------
%	FILE CONTENTS ENVIRONMENT
%----------------------------------------------------------------------------------------

% Usage:
% \begin{file}[optional filename, defaults to "File"]
%	File contents, for example, with a listings environment
% \end{file}

\mdfdefinestyle{file}{
	innertopmargin=1.6\baselineskip,
	innerbottommargin=0.8\baselineskip,
	topline=false, bottomline=false,
	leftline=false, rightline=false,
	leftmargin=2cm,
	rightmargin=2cm,
	singleextra={%
		\draw[fill=black!10!white](P)++(0,-1.2em)rectangle(P-|O);
		\node[anchor=north west]
		at(P-|O){\ttfamily\mdfilename};
		%
		\def\l{3em}
		\draw(O-|P)++(-\l,0)--++(\l,\l)--(P)--(P-|O)--(O)--cycle;
		\draw(O-|P)++(-\l,0)--++(0,\l)--++(\l,0);
	},
	nobreak,
}

% Define a custom environment for file contents
\newenvironment{file}[1][File]{ % Set the default filename to "File"
	\medskip
	\newcommand{\mdfilename}{#1}
	\begin{mdframed}[style=file]
}{
	\end{mdframed}
	\medskip
}

%----------------------------------------------------------------------------------------
%	NUMBERED QUESTIONS ENVIRONMENT
%----------------------------------------------------------------------------------------

% Usage:
% \begin{question}[optional title]
%	Question contents
% \end{question}

\mdfdefinestyle{question}{
	innertopmargin=1.2\baselineskip,
	innerbottommargin=0.8\baselineskip,
	roundcorner=5pt,
	nobreak,
	singleextra={%
		\draw(P-|O)node[xshift=1em,anchor=west,fill=white,draw,rounded corners=5pt]{%
		Question \theQuestion\questionTitle};
	},
}

\newcounter{Question} % Stores the current question number that gets iterated with each new question

% Define a custom environment for numbered questions
\newenvironment{question}[1][\unskip]{
	\bigskip
	\stepcounter{Question}
	\newcommand{\questionTitle}{~#1}
	\begin{mdframed}[style=question]
}{
	\end{mdframed}
	\medskip
}

%----------------------------------------------------------------------------------------
%	WARNING TEXT ENVIRONMENT
%----------------------------------------------------------------------------------------

% Usage:
% \begin{warn}[optional title, defaults to "Warning:"]
%	Contents
% \end{warn}

\mdfdefinestyle{warning}{
	topline=false, bottomline=false,
	leftline=false, rightline=false,
	nobreak,
	singleextra={%
		\draw(P-|O)++(-0.5em,0)node(tmp1){};
		\draw(P-|O)++(0.5em,0)node(tmp2){};
		\fill[black,rotate around={45:(P-|O)}](tmp1)rectangle(tmp2);
		\node at(P-|O){\color{white}\scriptsize\bf !};
		\draw[very thick](P-|O)++(0,-1em)--(O);%--(O-|P);
	}
}

% Define a custom environment for warning text
\newenvironment{warn}[1][Warning:]{ % Set the default warning to "Warning:"
	\medskip
	\begin{mdframed}[style=warning]
		\noindent{\textbf{#1}}
}{
	\end{mdframed}
}

%----------------------------------------------------------------------------------------
%	INFORMATION ENVIRONMENT
%----------------------------------------------------------------------------------------

% Usage:
% \begin{info}[optional title, defaults to "Info:"]
% 	contents
% 	\end{info}

\mdfdefinestyle{info}{%
	topline=false, bottomline=false,
	leftline=false, rightline=false,
	nobreak,
	singleextra={%
		\fill[black](P-|O)circle[radius=0.4em];
		\node at(P-|O){\color{white}\scriptsize\bf i};
		\draw[very thick](P-|O)++(0,-0.8em)--(O);%--(O-|P);
	}
}

% Define a custom environment for information
\newenvironment{info}[1][Info:]{ % Set the default title to "Info:"
	\medskip
	\begin{mdframed}[style=info]
		\noindent{\textbf{#1}}
}{
	\end{mdframed}
}

% Disable paragraph indentation
%\setlength{\parindent}{0pt}
 % Include the file specifying the document structure and custom commands

%----------------------------------------------------------------------------------------
%	ASSIGNMENT INFORMATION
%----------------------------------------------------------------------------------------

\title{Esercizi Blocco 1} % Title of the assignment

\author{Luca Oliveri\\ \texttt{olivieri.luca@outlook.com}} % Author name and email address

\date{Università di Trento --- \today} % University, school and/or department name(s) and a date

%----------------------------------------------------------------------------------------

\begin{document}

\maketitle % Print the title

%----------------------------------------------------------------------------------------
%	INTRODUCTION
%----------------------------------------------------------------------------------------

\section*{Introduzione} % Unnumbered section
Abbreviamo "scrivere un programma" con SUP. Questo blocco prevede l'uso di funzioni. Non è esplicitamente indicato quali porzioni di codice sono da delegare ad una funzione.


\setcounter{section}{1}

%------------------------------------------------

\subsection{} 
SUP che si fa dare due numeri interi $x$ e $y$ e calcola $x^y$. Fare due versioni, una che per il loop interno usa \texttt{for}, l'altra che usa \texttt{while}. Non usare \texttt{pow()}, questo perché se andate a leggere la documentazione, vedrete che prende in ingresso due \texttt{double}, significa quindi che al suo interno non compie un elevamento a potenza tra interi, ma tra numeri reali. Pur permettendoci di ottenere lo stesso risultato, questo implica che l'esecuzione sarà ben più lenta e inoltre potremmo imbatterci in errori numerici.


\subsection{}
SUP che dato un numero intero e un numero indice, individua la cifra alla posizione indice del primo numero inserito. Esempio: 39842, indice 2 (decine) = 4 decine. Farsi stampare solo la cifra, non serve il nome posizionale (decine, centinaia, ecc).
\begin{info} 
	Per scomporre il numero in cifre, usare l'operatore divisione e resto con le potenze di 10. Questa è un'operazione che capita spesso in programmazione. Svolgere prima l'esercizio precedente: si può riutilizzare del codice.
\end{info}

\subsection{}
Una parola, frase o numero palindromo ha la proprietà che può essere letto in entrambi i sensi. Per esempio i seguenti numeri da 5 cifre sono palindromi: 12321, 55555, 85658, 11611. SUP che legge un numero a 5 cifre e determina se questo è palindromo o meno.
\begin{info} 
	Si consiglia di svolgere prima i due esercizi precedenti.
\end{info}
\begin{warn} [Estensione:]
	Il numero in ingresso non ha lunghezza fissata a 5 ma ha lunghezza variabile. Con questa estensione diventa l'esercizio algoritmicamente più complesso di questo Blocco.
\end{warn}

\subsection{}
SUP che stampa senza sosta i multipli di due. Quindi 2, 4, 8, 16, eccetera. Il loop è infinito. Cosa succede e perché?

\subsection{}
Il fattoriale di un numero non-negativo $n$ si scrive $n!$. Ed è definito come segue:
$$n! = n \cdot (n - 1) \cdot (n - 2) \cdot (n - 3) \cdot \ldots \cdot 1$$
Se $n==0$ allora $0! = 1$.
Per esempio $$5! = 5 \cdot 4 \cdot 3 \cdot 2 \cdot 1 = 120$$
SUP che legge un numero non-negativo e calcolarne il fattoriale. Fino a che numero si ottengono risultati corretti? Controllare che l'utente inserisca un numero adeguato e in caso contrario scrivere un messaggio e terminare l'esecuzione. 
\begin{info} 
	\href{https://en.wikipedia.org/wiki/Recursion_(computer_science)#Recursion_versus_iteration}{È dimostrato} che algoritmi scritti usando la ricorsione possono essere riscritti usando cicli iterativi classici (\texttt{while}/\texttt{for}) e viceversa. Provare a scrivere entrambe le soluzioni.
\end{info}


\subsection{}
SUP che prende in ingresso un intero e calcola il valore della costante di eulero $e$ con la seguente formula:
$$e = 1 + \frac{1}{1!} + \frac{1}{2!} + \frac{1}{3!} + \cdots$$
Usare il valore in ingresso per quanto a fondo esplorare la progressione. Confrontare il valore con la funzione \texttt{exp()} presente in \texttt{math.h} con argomento 1. Se avete dubbi sulle funzioni di libreria, Google è \href{http://www.cplusplus.com/reference/cmath/exp/}{tuo} \href{https://www.geeksforgeeks.org/exp-function-cpp/}{amico}. Quante iterazioni sono necessarie per far scendere l'errore sotto ai millesimi?


\section*{Esercizi \texttt{CodeStepByStep}}
\begin{itemize}
	\item \href{https://www.codestepbystep.com/problem/view/cpp/parameters/BMI}{BMI}
	\item \href{https://www.codestepbystep.com/problem/view/cpp/parameters/boxOfStars}{boxOfStars}
	\item \href{https://www.codestepbystep.com/problem/view/cpp/parameters/Projectile}{Projectile}
\end{itemize}



\end{document}
