%%%%%%%%%%%%%%%%%%%%%%%%%%%%%%%%%%%%%%%%%
% Lachaise Assignment
% LaTeX Template
% Version 1.0 (26/6/2018)
%
% This template originates from:
% http://www.LaTeXTemplates.com
%
% Authors:
% Marion Lachaise & François Févotte
% Vel (vel@LaTeXTemplates.com)
%
% License:
% CC BY-NC-SA 3.0 (http://creativecommons.org/licenses/by-nc-sa/3.0/)
% 
%%%%%%%%%%%%%%%%%%%%%%%%%%%%%%%%%%%%%%%%%

%----------------------------------------------------------------------------------------
%	PACKAGES AND OTHER DOCUMENT CONFIGURATIONS
%----------------------------------------------------------------------------------------

\documentclass{article}

\input{../../structure.tex} % Include the file specifying the document structure and custom commands

%----------------------------------------------------------------------------------------
%	ASSIGNMENT INFORMATION
%----------------------------------------------------------------------------------------

\title{Esercizi Blocco 0} % Title of the assignment

\author{Luca Oliveri\\ \texttt{olivieri.luca@outlook.com}} % Author name and email address

\date{Università di Trento --- \today} % University, school and/or department name(s) and a date

%----------------------------------------------------------------------------------------

\begin{document}

\maketitle % Print the title

%----------------------------------------------------------------------------------------
%	INTRODUCTION
%----------------------------------------------------------------------------------------

\section*{Introduzione} % Unnumbered section
Questa serie di file di esercizi verrà pubblicata mano a meno che progrediremo nel corso. Gli esercizi non saranno né valutati né corretti. Nonostante ciò, sono da considerarsi chiave per preparazione all'esame.


\begin{info} 
	Molti degli esercizi sono proposti dal sito \href{https://www.codestepbystep.com/problem/list/cpp}{\texttt{CodeStepByStep}}. Previa registrazione, il sito propone un box dove si può direttamente scrivere il codice e poi eseguire un test automatico della sua correttezza. Il mio consiglio è, quando possibile, quello di sviluppare su VSCode sfruttando il sistema dei sotto-progetti, uno per esercizio. Copiare poi il codice su \texttt{CodeStepByStep} per verificarne la correttezza solo alla fine. 
\end{info}


\setcounter{section}{0}

%------------------------------------------------

\subsection{} 
Scrivere un programma completo che chiede all'utente tre numeri interi e ne calcola il prodotto. 

\begin{warn}[Occhio:]Se invece venisse chiesto di chiedere dieci numeri?
\end{warn}


\subsection{}
Scrivere un programma che chide all'utente due numeri e ne calcola somma, prodotto, differenza, divisione e resto.


\subsection{}
Scrivere un programma che scrive i numeri da 1 a 4 sulla stessa linea. Ripetere tre volte lo stesso output utilizzando i seguenti metodi.
\begin{itemize}
	\item Usando una sola volta \texttt{cout} e separando gli spazi dai numeri Es: \texttt{"3"} e \texttt{" "}.
	\item Usando una sola volta \texttt{cout} mantenendo gli spazi insieme ai numeri Es: \texttt{"3 "}.
	\item Usando quattro \texttt{cout} separati.
\end{itemize}


\vspace{-10pt}
\begin{warn}[Estensione:]
	L'utente inserisce un numero e vengono stampati i numeri da 1 al numero inserito. Provate a capire quale tra le prime due opzioni è più efficiente.
\end{warn}

\subsection{}
Scrivere un programma che chide all'utente di inserire due interi, e poi scrive il numero più grande seguito dalla string \texttt{" is larger"}. Se i due numeri sono uguali, scrivi il messaggio \texttt{" These numbers are equal"}.

\subsection{}
Scrivere un programma che legge tre numeri e calcola la somma, la media, il prodotto, il più grande e il più piccolo.

\subsection{}
Scrivere un programma che si fa inserire la lunghezza del raggio e calcola il diametro, circonferenza e area. Ci sono diversi modi nell'ottenere la costante $\pi$, tra cui scriversela a mano. Fare i calcoli direttamente nei comandi di \texttt{cout}.

\subsection{}
Scrivere un programma che prende un intero e verifica se questo è pari o dispari e scrive qualcosa a proposito.

\subsection{}
Scrivere un programma che prende due numeri interi e determina se il primo è un multiplo del secondo.

\subsection{}
Scrivere un programma che stampa la tabella ASCII in questo formato, limitandosi alle lettere maiuscole, minuscole e le dieci cifre:

\texttt{CHAR: A ASCII: 65}

\texttt{CHAR: B ASCII: 66}

\texttt{CHAR: C ASCII: 67}


\begin{info}
	Dato un intero \texttt{i}, il su corrispondente carattere ASCII può essere ricavato tramite l'operatore di cast, quindi:
	\texttt{cout {<}< (char) 65} stamperà \texttt{A}. 
	Inoltre ricordare che invece scrivendo \texttt{int i = 'A'} il valore di \texttt{i} sarà 65. 
\end{info}


\section*{Esercizi \texttt{CodeStepByStep}}
\begin{itemize}
	\item \href{https://www.codestepbystep.com/problem/view/cpp/ifelse/ifElseMystery1}{\texttt{ifElseMystery1}}
	\item \href{https://www.codestepbystep.com/problem/view/cpp/ifelse/ifElseMystery2}{\texttt{ifElseMystery2}}
	\item \href{https://www.codestepbystep.com/problem/view/cpp/ifelse/percentageGrade}{\texttt{percentageGrade}}
	\item \href{https://www.codestepbystep.com/problem/view/cpp/basics/evenAverage}{\texttt{evenAverage}}
	\item \href{https://www.codestepbystep.com/problem/view/cpp/basics/fixErrors}{\texttt{fixErrors}}
	\item \href{https://www.codestepbystep.com/problem/view/cpp/basics/numberSquare}{\texttt{numberSquare}}
\end{itemize}



\end{document}
