%%%%%%%%%%%%%%%%%%%%%%%%%%%%%%%%%%%%%%%%%
% Lachaise Assignment
% LaTeX Template
% Version 1.0 (26/6/2018)
%
% This template originates from:
% http://www.LaTeXTemplates.com
%
% Authors:
% Marion Lachaise & François Févotte
% Vel (vel@LaTeXTemplates.com)
%
% License:
% CC BY-NC-SA 3.0 (http://creativecommons.org/licenses/by-nc-sa/3.0/)
% 
%%%%%%%%%%%%%%%%%%%%%%%%%%%%%%%%%%%%%%%%%

%----------------------------------------------------------------------------------------
%	PACKAGES AND OTHER DOCUMENT CONFIGURATIONS
%----------------------------------------------------------------------------------------

\documentclass{article}

\input{structure.tex} % Include the file specifying the document structure and custom commands

%----------------------------------------------------------------------------------------
%	ASSIGNMENT INFORMATION
%----------------------------------------------------------------------------------------

\title{Esercizi Blocco 1} % Title of the assignment

\author{Luca Oliveri\\ \texttt{luca.olivieri-1@unitn.it}} % Author name and email address

\date{Università di Trento --- \today} % University, school and/or department name(s) and a date

%----------------------------------------------------------------------------------------

\begin{document}

\maketitle % Print the title

%----------------------------------------------------------------------------------------
%	INTRODUCTION
%----------------------------------------------------------------------------------------

\section*{Introduzione} % Unnumbered section
Abbreviamo "scrivere un programma" con SUP.


% \begin{info} 
% 	Molti degli esercizi sono proposti dal sito \href{https://www.codestepbystep.com/problem/list/cpp}{\texttt{CodeStepByStep}}. Previa registrazione, il sito propone un box dove si può direttamente scrivere il codice e poi eseguire un test automatico della sua correttezza. Il mio consiglio è, quando possibile, quello di sviluppare su VSCode sfruttando il sistema dei sotto-progetti, uno per esercizio. Copiare poi il codice su \texttt{CodeStepByStep} per verificarne la correttezza solo alla fine. 
% \end{info}


\setcounter{section}{1}

%------------------------------------------------

\subsection{} 
SUP che si fa dare due numeri interi $x$ e $y$ e calcola $x^y$. Fare due versioni, una che per il loop interno usa \texttt{for}, l'altra che usa \texttt{while}. Non usare \texttt{pow()}.

\subsection{}
Trovare il numero più grande in un array è un problema ricorrente in programmazione. SUP che dati 10 numeri, inseriti dall'utente, restituisce il numero più grande.

\subsection{}
Modificare il programma precedente in modo che trovi i due numeri più grandi.

\subsection{}
Una parola, frase o numero palindromo ha la proprietà che può essere letto in entrambi i sensi. Per esempio i seguenti numeri da 5 cifre sono palindromi: 12321, 55555, 85658, 11611. SUP che legge un numero a 5 cifre e determina se questo è palindromo o meno.
\begin{info} 
	Usare l'operatore divisione e resto per dividere il numero in cifre. Anche questa è un'operazione che capita spesso di fare in programmazione.
\end{info}
\vspace{-10pt}
\begin{warn}[Estensione non banale:]
	L'utente inserisce un numero di lunghezza a suo piacimento. 
\end{warn}

\subsection{}
SUP che stampa senza sosta i multipli di due. Quindi 2, 4, 8, 16, eccetera. Il loop è infinito. Cosa succede e perché?

\subsection{}
Il fattoriale di un numero non-negativo $n$ si scrive $n!$. Ed è definito come segue:
$$n! = n \cdot (n - 1) \cdot (n - 2) \cdot (n - 3) \cdot \ldots \cdot 1$$
Se $n==0$ allora $0! = 1$.
Per esempio $$5! = 5 \cdot 4 \cdot 3 \cdot 2 \cdot 1 = 120$$
SUP che legge un numero non-negativo e calcolarne il fattoriale. Fino a che numero si ottengono risultati corretti? Controllare che l'utente inserisca un numero adeguato e in caso contrario scrivere un messaggio e terminare l'esecuzione. 

\subsection{}
SUP che prende in ingresso un intero e calcola il valore della costante di eulero $e$ con la seguente formula:
$$e = 1 + \frac{1}{1!} + \frac{1}{2!} + \frac{1}{3!} + \cdots$$
Usare il valore in ingresso per quanto a fondo esplorare la progressione. Confrontare il valore con la funzione \texttt{exp()} presente in \texttt{math.h} con argomento 1. Se avete dubbi sulle funzioni di libreria, Google è \href{http://www.cplusplus.com/reference/cmath/exp/}{tuo} \href{https://www.geeksforgeeks.org/exp-function-cpp/}{amico}.

\subsection{}
SUP che legge 10 numeri dall'utente e ne calcola la somma. Fermarsi a leggere numeri quando l'utente inserisce tutti i 10 numeri oppure quando inserisce 0.

\subsection{}
SUP che prende 10 numeri e calcola la somma solamente dei numeri pari.

\subsection{}
SUP che calcola la media dei voti di un esame. Il programma si fa dare prima il numero totale di studenti, poi i voti. Supporre un limite massimo di studenti. 

\begin{info} 
	Usare l'operatore divisione e resto per dividere il numero in cifre. Anche questa è un'operazione che capita spesso di fare in programmazione.
\end{info}


\section*{Esercizi \texttt{CodeStepByStep}}
\begin{itemize}
	\item \href{https://www.codestepbystep.com/problem/view/cpp/ifelse/ifElseMystery1}{ifElseMystery1}
	\item \href{https://www.codestepbystep.com/problem/view/cpp/ifelse/ifElseMystery2}{ifElseMystery2}
	\item \href{https://www.codestepbystep.com/problem/view/cpp/ifelse/percentageGrade}{percentageGrade}
	\item \href{https://www.codestepbystep.com/problem/view/cpp/basics/evenAverage}{evenAverage}
	\item \href{https://www.codestepbystep.com/problem/view/cpp/basics/fixErrors}{fixErrors}
	\item \href{https://www.codestepbystep.com/problem/view/cpp/basics/numberSquare}{numberSquare}
\end{itemize}



\end{document}
